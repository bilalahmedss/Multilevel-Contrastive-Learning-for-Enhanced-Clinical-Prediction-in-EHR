\documentclass[a4paper]{article}
\usepackage{geometry}
\geometry{a4paper, margin=1in}
\usepackage{graphicx}
\usepackage{hyperref}
\begin{document}
    \begin{center}
        \includegraphics[width=10cm]{LOGOHABIB.png}
        \\
        \vspace{5mm}
        {\huge \textbf{Computer Vision}} \\ % Title
        \vspace{7mm}
        {\LARGE \textbf{Detailed Literature Review}} \\ % Subtitle
        \vspace{15mm}
        {\Large \textbf{``Multilevel Contrastive Learning for Enhanced Clinical Prediction in Electronic Health Records''}}\\
        \vspace{70mm}
        \begin{minipage}{0.7\textwidth}
            \begin{tabular}{rlr}
                {\large \textbf{Name:}}            & {\large Bilal Ahmed}       & {\large (ba08018)} \\[2mm]
                {\large \textbf{Class:}}           & {\large 2026}              &                    \\[3mm]
                {\large \textbf{Instructor:}}      & {\large Muhammad Farhan}   &                    \\[3mm]
                {\large \textbf{Submission Date:}} & {\large 6th December 2024} &
            \end{tabular}
        \end{minipage}
        \vfill % Center the title content vertically
    \end{center}
    \newpage
    \tableofcontents
    \clearpage

    \section{Introduction}

    \subsection{Purpose and Scope}
    The purpose of this review is to examine advancements in contrastive learning
    for clinical predictions using Electronic Health Records (EHRs), focusing on
    three key articles: FairEHR-CLP, Contrastive Learning-based Imputation-Prediction
    Networks, and SCEHR. The scope includes the application of contrastive
    learning to improve accuracy, fairness, and data imputation in clinical
    prediction tasks. These studies highlight methodologies for leveraging multimodal
    data and mitigating biases, critical for developing enhanced predictive
    models.

    \subsection{Relevance}
    The reviewed studies are directly relevant to our project titled \textit{Multilevel
    Contrastive Learning for Enhanced Clinical Prediction in Electronic Health
    Records}. They provide insights into applying contrastive learning
    techniques to address data heterogeneity, fairness concerns, and predictive
    accuracy and it challenges central to modern EHR-based clinical prediction
    models.

    \subsection{Organization}
    The review begins by outlining the search and selection criteria used to
    identify the three articles, followed by a thematic and methodological discussion.
    A critical analysis highlights key findings, strengths, and limitations. The
    relevance of these works to the research question is established, and the
    synthesis section integrates these findings into broader trends. Finally,
    the review concludes with a summary and identification of gaps for future
    work.

    \section{Search and Selection Criteria}

    \subsection{Search Strategy}
    A targeted search was conducted on academic databases such as arXiv and PubMed.
    Keywords included "contrastive learning," "clinical prediction," "multimodal
    EHR," and "fairness in AI." Articles published between 2021 and 2024 were
    prioritized to ensure the inclusion of recent advancements.

    \subsection{Inclusion/Exclusion Criteria}
    \textbf{Inclusion Criteria:}
    \begin{itemize}
        \item Studies employing contrastive learning for clinical prediction tasks.

        \item Research incorporating multimodal EHR data.

        \item Articles addressing fairness or data imputation in predictive models.
    \end{itemize}
    \textbf{Exclusion Criteria:}
    \begin{itemize}
        \item Studies focused solely on unimodal data.

        \item Papers lacking a rigorous methodological framework or experimental
            validation.
    \end{itemize}

    \subsection{Scope of Literature}
    The selected articles explore three critical areas: fairness-aware
    contrastive learning (FairEHR-CLP), data imputation for mortality risk prediction
    (Contrastive Learning-based Imputation-Prediction Networks), and supervised
    contrastive learning for longitudinal EHR data (SCEHR).

    \section{Thematic or Chronological Organization}

    \subsection{Chronological}
    The articles span from 2021 to 2024, reflecting the evolution of contrastive
    learning methodologies in healthcare:
    \begin{itemize}
        \item SCEHR (2021) introduces supervised contrastive learning for EHR-based
            risk prediction.

        \item Contrastive Learning-based Imputation-Prediction Networks (2023) focuses
            on addressing data irregularities and missing values in EHR datasets.

        \item FairEHR-CLP (2024) emphasizes fairness-aware contrastive learning across
            multimodal EHR data.
    \end{itemize}

    \subsection{Thematic}
    Thematically, the studies can be categorized as:
    \begin{itemize}
        \item \textbf{Fairness in Clinical Predictions:} FairEHR-CLP addresses
            demographic bias and ensures equitable outcomes.

        \item \textbf{Data Imputation:} The Imputation-Prediction framework
            integrates imputation with prediction tasks to handle missing data.

        \item \textbf{Representation Learning:} SCEHR focuses on representation
            learning for longitudinal EHR data using supervised contrastive
            regularization.
    \end{itemize}

    \subsection{Methodological}
    All three articles use contrastive learning, but with distinct
    methodological focuses:
    \begin{itemize}
        \item FairEHR-CLP uses counterfactual patient generation to mitigate demographic
            biases.

        \item The Imputation-Prediction Network integrates graph-based patient stratification
            for data imputation.

        \item SCEHR introduces a composite loss function to address data imbalance
            in classification tasks.
    \end{itemize}

    \section{Critical Analysis}

    \subsection{Summarize}
    FairEHR-CLP provides a fairness-aware framework by generating counterfactual
    patient representations to align embeddings across demographic attributes while
    preserving clinical information. Using a fairness-aware loss function, it mitigates
    demographic bias, ensuring equitable outcomes for sensitive groups such as
    gender and ethnicity, making it impactful for fair healthcare delivery.

    The Imputation-Prediction framework tackles missing data and irregular time intervals
    in EHR datasets by leveraging graph-based patient stratification to enhance imputation
    accuracy. Contrastive learning improves patient data representation, leading
    to reliable and accurate mortality risk predictions, offering a dual focus
    on imputation and prediction.

    SCEHR applies supervised contrastive learning for longitudinal EHR data, introducing
    a composite loss function to address data imbalance. This framework improves
    classification tasks such as in-hospital mortality prediction and patient
    phenotyping, validated across multiple datasets.

    \subsection{Compare and Contrast}
    FairEHR-CLP emphasizes fairness by addressing demographic biases and ensuring
    equitable predictions across different patient groups. Its primary
    innovation lies in the generation of counterfactual patient representations,
    making it unique among the three frameworks. However, its focus on fairness means
    that it does not address other challenges such as data imputation or irregular
    time intervals.

    The Imputation-Prediction framework focuses on data quality, specifically
    missing values and irregular time intervals, which are major challenges in EHR
    datasets. Unlike FairEHR-CLP, it does not address fairness explicitly but instead
    improves predictive accuracy by leveraging patient stratification and graph-based
    methodologies. This framework bridges the gap between imputation and
    prediction, enabling a comprehensive approach to handling data irregularities.

    SCEHR takes a different approach by addressing data imbalance through supervised
    contrastive learning. Unlike the other two frameworks, SCEHR focuses on binary
    and multi-label classification tasks, primarily improving representation
    learning in longitudinal EHR data. While it effectively enhances predictive performance,
    it lacks the fairness and data quality aspects emphasized in FairEHR-CLP and
    the Imputation-Prediction framework.

    Overall, all three frameworks leverage contrastive learning but apply it to distinct
    challenges—fairness (FairEHR-CLP), data quality (Imputation-Prediction Network),
    and representation learning in imbalanced datasets (SCEHR). These
    differences highlight the versatility of contrastive learning in addressing
    diverse challenges in clinical prediction.

    \subsection{Strengths and Weaknesses}
    \textbf{Strengths:}
    \begin{itemize}
        \item \textbf{FairEHR-CLP:}
            \begin{itemize}
                \item Innovative use of counterfactual patient generation to address
                    demographic bias.

                \item Robust fairness-aware loss function combining contrastive learning
                    with demographic regularization.

                \item Demonstrated effectiveness in mitigating biases across sensitive
                    attributes such as gender and ethnicity.
            \end{itemize}

        \item \textbf{Imputation-Prediction Network:}
            \begin{itemize}
                \item Novel integration of imputation and prediction tasks,
                    addressing missing data and irregular time intervals.

                \item Graph-based patient stratification enhances the quality of
                    imputed data.

                \item Improves predictive accuracy by leveraging contrastive learning
                    for better patient representation.
            \end{itemize}

        \item \textbf{SCEHR:}
            \begin{itemize}
                \item Effective handling of imbalanced data through supervised contrastive
                    loss functions.

                \item Demonstrated scalability and performance on multiple clinical
                    datasets.

                \item Improves binary and multi-label classification tasks in longitudinal
                    EHR data.
            \end{itemize}
    \end{itemize}

    \textbf{Weaknesses:}
    \begin{itemize}
        \item \textbf{FairEHR-CLP:}
            \begin{itemize}
                \item Limited to addressing demographic biases; does not account
                    for institutional or systemic biases.

                \item Requires validation on more diverse datasets to ensure generalizability.
            \end{itemize}

        \item \textbf{Imputation-Prediction Network:}
            \begin{itemize}
                \item Computationally expensive due to graph-based patient stratification.

                \item Limited scalability for very large datasets with high levels
                    of missingness.
            \end{itemize}

        \item \textbf{SCEHR:}
            \begin{itemize}
                \item Focused primarily on classification tasks, lacking
                    integration with fairness-aware or imputation-focused
                    methods.

                \item Does not explicitly address challenges associated with multimodal
                    EHR data.
            \end{itemize}
    \end{itemize}

    \subsection{Identify Gaps}
    While these frameworks provide significant advancements, they also reveal gaps
    in current research. FairEHR-CLP is limited to addressing demographic biases
    and does not consider other types of biases, such as those stemming from institutional
    or systemic factors. The Imputation-Prediction framework, while innovative,
    lacks fairness considerations and may face scalability challenges for large-scale
    datasets. SCEHR effectively handles data imbalance but does not integrate multimodal
    data or address fairness and data quality comprehensively. Future research
    should focus on developing unified frameworks that integrate fairness, data imputation,
    and multimodal learning while ensuring scalability and generalizability
    across diverse healthcare settings.

    \section{Relevance to Your Research}

    \subsection{Link to Research Question}
    These studies provide foundational insights into applying multilevel contrastive
    learning for clinical prediction, directly supporting the goals of this
    project.

    \subsection{Conceptual Framework}
    The FairEHR-CLP framework's fairness-aware loss functions and counterfactual
    generation serve as a conceptual basis for multilevel approaches. The Imputation-Prediction
    Network’s handling of data irregularities offers inspiration for integrating
    imputation tasks into multilevel learning.

    \section{Synthesis}

    \subsection{Integration}
    The reviewed articles demonstrate the integration of contrastive learning
    with fairness-aware, imputation-focused, and representation learning
    frameworks to address unique challenges in clinical predictions.

    \subsection{Trend Analysis}
    A clear trend is observed toward proactive bias mitigation, robust data representation,
    and multimodal integration using contrastive learning techniques in
    healthcare.

    \section{Citations and References}

    \subsection{Comprehensive Referencing}
    \begin{itemize}
        \item Wang, Y., Pillai, M., Zhao, Y., Curtin, C., \& Hernandez-Boussard,
            T. (2024). FairEHR-CLP: Towards Fairness-Aware Clinical Predictions with
            Contrastive Learning in Multimodal Electronic Health Records.
            \textit{arXiv preprint}. Retrieved from
            \url{https://arxiv.org/abs/2402.00955}

        \item Liu, Y., Zhang, Z., Qin, S., Salim, F. D., \& Yepes, A. J. (2023).
            Contrastive Learning-based Imputation-Prediction Networks for In-hospital
            Mortality Risk Modeling using EHRs. \textit{arXiv preprint}. Retrieved
            from \url{https://arxiv.org/abs/2308.09896}

        \item Zang, C., \& Wang, F. (2021). SCEHR: Supervised Contrastive
            Learning for Clinical Risk Prediction using Electronic Health
            Records. \textit{arXiv preprint}. Retrieved from \url{https://arxiv.org/abs/2110.04943}
    \end{itemize}

    \subsection{Acknowledgment of ChatGPT Use}
    The summaries and rephrased content for this literature review were
    generated and refined with the assistance of ChatGPT. ChatGPT was used to enhance
    understanding, streamline the summarization process, and ensure clarity in presenting
    complex methodologies. While the content reflects the author's
    interpretation, ChatGPT played a supportive role in articulating the analysis
    more effectively.
\end{document}